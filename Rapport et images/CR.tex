\documentclass{beamer}
\hypersetup{
    colorlinks=true,
    linkcolor=blue,
    filecolor=magenta,      
    urlcolor=cyan,
    pdftitle={Overleaf Example},
    pdfpagemode=FullScreen,
    }

\usepackage{caption}
\captionsetup{labelformat=empty, textfont=it}

% Chargement du logo
\logo{\includegraphics[height=0.75cm]{images/logo_enseirb.jpg}}

% Choix du thème
\usetheme{CambridgeUS}  % Thème de présentation
\usecolortheme{dolphin} % Thème de couleurs

% Informations sur la présentation
\title{Séance du 24/01}
\author{Projet S8: TRACK}
\institute[T2, Enseirb-Matmeca]{Département Télécommunication\\Enseirb-Matmeca, Bordeaux}
\date{\today}

% Début du document
\begin{document}

% Page de titre
\begin{frame}
  \titlepage
\end{frame}

% Introduction
\begin{frame}
  \frametitle{Ordre du Jour}
  \begin{itemize}
    \item Présentation des avancées
    \item Méthodes et Outils
    \item Présentation du calendrier
    \item Questions 
  \end{itemize}
\end{frame}

% Compte rendu des avancées
\begin{frame}
  \frametitle{Compte rendu des avancées}
  \begin{itemize}
    \item MRU :
    \begin{itemize}
        \item[\hspace{1cm}] Théorie implémentée sous Python
        \item[\hspace{1cm}] Premières trajectoires 
    \end{itemize}
    \item MUA et Singer
    \begin{itemize}
        \item[\hspace{1cm}] Calculs théoriques réalisés
        \item[\hspace{1cm}] Implémentation sous Python et génération de trajectoires
    \end{itemize}
    \item Création d'une bibliothèque de fonctions communes
    \begin{itemize}
        \item[\hspace{1cm}] Génération de coordonnées X et Y
        \item[\hspace{1cm}] Stockages des itérations de générations de vecteurs
        \item[\hspace{1cm}] Fonctions d'affichages des données
    \end{itemize}
  \end{itemize}
\end{frame}



% Calendrier de travail
\begin{frame}
  \frametitle{Calendrier de travail}
    \begin{itemize}
        \item Pour la semaine prochaine : 
        \begin{itemize}
            \item[\hspace{1cm}] Générer les bases de données 
            \item[\hspace{2cm}] ($\sim$ 50,000 échantillons | tailles variables ($100 < N < 750$))
        \end{itemize}
        \item La suite ?
    \end{itemize}
\end{frame}

% Questions
\begin{frame}
  \frametitle{Questions}
  \begin{itemize}
    \item Cohérence des courbes ?
    \item Quelle quantité d'échantillons ? Tailles variables des échantillons ?
    \item Rebruiter nos données générées ? 
    \item Un format de stockages de donnés à privilégier ?
  \end{itemize}
\end{frame}

% Fin de la présentation
\end{document}
