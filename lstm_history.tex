\documentclass{beamer}

\usepackage[utf8]{inputenc}
\usepackage{graphicx}

\title{Histoire du LSTM}
\author{}
\date{}

\begin{document}

\begin{frame}
    \titlepage
\end{frame}

% Slide 1: Introduction à l'histoire du LSTM
\begin{frame}{Origine des RNN et limites}
    \begin{itemize}
        \item Les réseaux de neurones récurrents (RNN) ont été développés dans les années 1980.
        \item Leur objectif : modéliser des séquences de données temporelles.
        \item Problèmes majeurs :
        \begin{itemize}
            \item Disparition et explosion du gradient.
            \item Difficulté à capturer des dépendances à long terme.
        \end{itemize}
    \end{itemize}
\end{frame}

% Slide 2: Naissance du LSTM
\begin{frame}{Naissance du LSTM (1997)}
    \begin{itemize}
        \item En 1997, \textbf{Sepp Hochreiter} et \textbf{Jürgen Schmidhuber} proposent le Long Short-Term Memory (LSTM).
        \item Solution aux limitations des RNN classiques grâce à un mécanisme de portes :
        \begin{itemize}
            \item Porte d’oubli : contrôle quelles informations doivent être effacées.
            \item Porte d’entrée : régule les nouvelles informations stockées.
            \item Porte de sortie : décide quelles informations transmettre à l’étape suivante.
        \end{itemize}
        \item Initialement peu utilisé, il devient populaire avec l’essor du deep learning et des capacités de calcul accrues.
    \end{itemize}
\end{frame}

\end{document}
